\section*{Übersicht}

Mit Hilfe des Mößbauereffekts lässt sich eine sehr genaue und hochauflösende Aufnahme von Energiespektren durchführen. Der Mößbauereffekt beschreibt die rückstoßfreie Absorption und Emission von Photonen und wird in diesem Versuch zur Untersuchung der Emissionslinien des $14,4\,\si{keV}$-Zustands von $^{57}Fe$ verwendet. Dabei wird die Lebensdauer des Zustands auf zwei unterschiedliche Arten bestimmt und die Hyperfeinaufspaltung von Natureisen vermessen.\\

Für die Lebensdauer des $14,4\,\si{keV}$-Zustands von $^{57}Fe$ ergeben sich dabei folgende Ergebnisse:
\begin{align*}
	\tau_1&=(110\pm70)\,\si{ns}\\
	\tau_2&=(162\pm12)\,\si{ns}\\
	\tau_\text{lit}&=141\,\si{ns}\text{.}
\end{align*}

Aus der Hyperfeinaufspaltung des Natureisens wird das magnetische Moment des Kerns und das Magnetfeld am Kernort bestimmt. Diese ergeben sich zu:
\begin{align*}
	\mu_a&=(-0,157\pm0,004)\,\mu_N\\
	\mu_\text{lit}&=(-0,15531\pm0,00004)\,\mu_N\\
	B&=(32,9\pm0,3)\,\si{T}\\
	B_\text{lit}&=(33,3\pm0,1)\,\si{T}\text{ .}
\end{align*}