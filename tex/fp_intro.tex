\section{Einleitung}
	
Der Mößbauereffekt beschreibt die rückstoßfreie Resonanzabsorption von Kernzuständen. Da bei diesem Prozess die Energieverluste eliminiert werden, die bisher häufig ein Hindernis darstellten und zugleich eine sehr hohe Auflösung und Genauigkeit erreicht werden kann, bot diese Entdeckung zahlreiche Anwendungen. Ein klassisches Beispiel ist hier die Spektroskopie. Es konnte mit Hilfe des Mößbauereffekts aber auch die Energieverschiebung von Photonen im Gravitationsfeld der Erde $\frac{\Delta E }{E} = 10^{-16} $ pro Meter nachgewiesen werden, die von der Allgemeinen Relativitätstheorie vorhergesagt wird. Sogenannte MIMOS\footnote{\textbf{MI}niaturized \textbf{MO}essbauer \textbf Spektrometer} werden in Marssonden zur Analyse von Marsgestein verwendet.
Für die Entdeckung des nach ihm benannten Mößbauereffekts wurde Rudolf Mößbauer im Jahr 1961 der Nobelpreis verliehen.\\

In diesem Versuch sollen Absorptionslinien im Einlinien-Absorber Edelstahl und im Sechslinien-Absorber Eisen gemessen werden und deren Lebensdauer bestimmt werden. $^{57}$Fe weist eine sehr schmale Linie bei einer Energie von $14,4\,\mathrm{keV}$ auf.
Zur Messung dieses Übergangs wird der Absorber auf einem beweglichen Schlitten, der sich mit Geschwindigkeiten im Bereich von wenigen $\si{\frac{mm}{s}}$ bewegt, von einer $^{57}$Co-Probe bestrahlt, in der durch den Zerfall von Cobalt in einen angeregten Zustand von $^{57}$Fe, der wiederum in den Grundzustand zerfällt, ein Photon der Energie $14,4\,\mathrm{keV}$ abgestrahlt wird. Ein Detektor bestehend aus Szintillator und Photomultiplier hinter dem Absorber nimmt die Zählrate in Abhängigkeit der Geschwindigkeit auf. Im so entstehenden Spektrum erwarten wir den typischen Mößbauerpeak zu sehen, aus dem die Lebensdauer des angeregten Zustands bestimmt werden kann, sowie aus der Verschiebung zur theoretischen Resonanzfrequenz die Isomerieverschiebung. Für Natureisen kann aus der Hyperfeinaufspaltung, die Grund für die sechs Absorptionslinien ist, das magnetische Feld bestimmt werden. Außerdem werden verschiedene Messungen zur Bestimmung des Untergrunds durchgeführt.
Der in diesem Versuch untersuchte Übergang wurde bereits im Versuch "`Kurze Halbwertszeiten"' mit der Methode der verzögerten Koinzidenzen gemessen. 


